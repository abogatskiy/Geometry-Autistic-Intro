
These notes started when I, along with my dear friend Umang Mehta, taught a series of seminars to our fellow physics grad students at UChicago in 2018/19. My desire to do so was based, in part, on the fact that another friend of mine, Sam Pramodh, taught me many of these topics in a similar fashion two years earlier.

In 2024 these notes started getting greatly expanded as part of my effort to continue deepening my understanding of these subjects, as well as to continue teaching my friends. The subtitle \emph{"an autistic introduction"} refers to the only style of teaching that I find fully satisfying: \emph{bottom up}. In short, this means that I try to apply the following principle as extensively as reasonably possible:
\begin{quote}
    No sophisticated construction is introduced until it is motivated enough to feel necessary.
\end{quote}
Teaching in this way inherently forces one to understand the objects being introduced much deeper, always keeping one's eyes on the ultimate goal. It also helps reproduce the sense of discovery while learning mathematics: instead of being ``passed down from the top of a mountain'', definitions and theorems come up as the most natural objects and statements that one should consider. 

This is why, even though most of the material in these notes is copied from external sources, it is heavily refactored and synthesized to provide a bottom up narrative that, to a maximum reasonable extent, moves from general to specific, adding new structure only when it becomes necessary. The "prize" on which we keep our eyes in this entire process is the classification of fiber bundles, which eventually forces us to learn about homology and cohomology, to which the entire next Part is dedicated. 

Nevertheless, there is no such thing as a \emph{full understanding} of any subject. One of the wonderful unique properties of mathematics is that one can \emph{always} dig deeper and find ever more ``fundamental'' objects in terms of which familiar results can be unified and reformulated. In other words, \emph{things are never understood in themselves, but only in relation to each other}. As such, one of the main goals of this book is not to speed to the finish line of, say, gauge theory. The goal is to give a relatively comprehensive introductions to a variety of geometric tools, exploring their connections to each other, looking at each one from multiple sides, and applying them to as many familiar and simple examples as possible. This is in contrast with too much of academic mathematical literature, which is overloaded with hard theory and neglects giving the reader a simple example that illustrates the approach.

Another side effect of this approach is that different sections of this book are essentially comprehensive introductions into subjects that are usually taught separately, without too much cross-talk, so that it is relatively easy to start reading ``from the middle''. For example, the theory of Lie groups is rarely included in differential geometry courses, but it was actually inseparably connected to the creation of modern geometry. The word ``geometry'' itself, at the end of the 19th century, usually referred to a certain set of properties shared by what we now call homogeneous spaces of Lie groups. Even though I dislike the strict chronological approach to teaching, the general historic progression of the development of geometric ideas was ``correct'' in that Lie groups provide the simplest \emph{models} for most  geometric structures. Thus, a deep understanding of the structure and geometry of Lie groups makes the development of the theory of connections much more natural and easier to digest, not to mention the endless examples easily computable ``by hand'' that are provided by Lie groups.

I hope that someone will find this material helpful. It is certainly a passion project for me, so I am eager to keep adding to it and correcting errors (which are, alas, undoubtedly numerous). Lastly, keep in mind that the contents are being updated regularly, some later sections are incomplete, and the last part is completely empty. Please share any typos you find via the GitHub page for this book:
\begin{center}
    \url{https://github.com/abogatskiy/Geometry-Autistic-Intro}.
\end{center}
If you feel like something needs to be added to the book, you are welcome to message me with a request. Moreover, if you feel inspired to contribute to the book yourself, then you can fork the GitHub repo!