%\setkomafont{disposition}{\normalfont}  %KOMA use serif font for sectioning
\usepackage[T1]{fontenc}
\usepackage[latin9]{inputenc}
\usepackage{anyfontsize}
\usepackage{geometry}
\geometry{verbose,tmargin=2cm,bmargin=2.5cm,lmargin=1cm,rmargin=1cm,heightrounded}
\setlength{\parskip}{\smallskipamount}
\setlength{\parindent}{0pt}
\usepackage{color}
\usepackage{babel}
\usepackage{amsmath}
\usepackage{amsthm}
%\usepackage[amsthm]{ntheorem}
\usepackage{amssymb}
\usepackage{mathtools}
\usepackage{wasysym}
\usepackage{cancel}
\usepackage{imakeidx}
\usepackage{booktabs}
\usepackage{wrapfig}
\usepackage{adjustbox}
% \usepackage{breqn}

\usepackage{stmaryrd}
\expandafter\def\csname opt@stmaryrd.sty\endcsname{only,shortleftarrow,shortrightarrow} %fake options for stmaryd so that loading extpfeil later doesn't cause a clash
\usepackage{extpfeil} % lets define multiple long arrows
\newextarrow{\toto}{{20}{20}{20}{20}}{\bigRelbar\bigRelbar{\bigtwoarrowsleft\rightarrow\rightarrow}}

   
\usepackage[stable,bottom]{footmisc}
\interfootnotelinepenalty=10000
\makeindex
\usepackage[footnotesize]{caption}
\captionsetup{width=0.8\textwidth}
\DeclareCaptionLabelFormat{bf-parens}{\textbf{Fig. #2}}
\captionsetup{labelformat=bf-parens}

\usepackage{comment}
% \includecomment{comment}
\excludecomment{comment}

%%%%%% FONTS
%\usepackage[onlytext,lf]{MinionPro}      % enable all this to get custom fonts
%\usepackage[amsbb,subscriptcorrection,mtpcal,mtphrb]{mtpro2}
%OR
%\usepackage{pscyr}
%\usepackage[libertine]{newtxmath}
%OR
\usepackage{newtxtext}
\usepackage[varvw]{newtxmath}

% This fixes math in headings not becoming bold 
% From https://tex.stackexchange.com/questions/41379/automatically-typeset-math-in-section-headings-in-bold-face
\makeatletter
\g@addto@macro\bfseries{\boldmath}
\makeatother


\usepackage{tikz-cd}
%\usepackage{marginnote}

\usepackage[activate={true},step=1,tracking=true,kerning=true,spacing=false,factor=1100,stretch=20,shrink=20]{microtype}%text appearance improvements
\UseMicrotypeSet[protrusion]{basictext}
%\microtypecontext{spacing=nonfrench}
%\DisableLigatures[N]{encoding=*, family =*}

\usepackage{titlesec} %change title spacing for sections
\titlespacing*{\section}
{0pt}{0.7\baselineskip}{0.3\baselineskip}
\titlespacing*{\subsection}
{0pt}{0.7\baselineskip}{0.3\baselineskip}


%%%%%%%
\usepackage[unicode=true,pdfusetitle,
 bookmarks=true,bookmarksnumbered=false,bookmarksopen=false,
 breaklinks=false,pdfborder={0 0 1},backref=false,colorlinks=true]
 {hyperref}

 
\usepackage[nottoc]{tocbibind} % list bibiliography and other secondary sections in TOC
\usepackage{tocloft} % needed for the next line
\setlength\cftsubsecnumwidth{4em} % increase the space given to numbers in TOC


\usepackage{enumitem}
\setlist[itemize]{leftmargin=2em}
\setlist[enumerate]{leftmargin=2em,topsep=0pt,itemsep=-1ex,partopsep=1ex,parsep=1ex}


\makeatletter


%%%%%%%%%%%%%%%%%%%%%%%%%%%%%%%%%%%%%%%%%%%%%%%%%%%
%%%%%%%%%%% theorem envs %%%%%%%%%%%%%%%%

\flushbottom
\renewcommand\qedsymbol{$\blacksquare$}
\providecommand{\definitionname}{Definition}
\providecommand{\examplename}{Example}
\providecommand{\exercisename}{Exercise}
\providecommand{\propositionname}{Proposition}
\providecommand{\remarkname}{Remark}
\providecommand{\theoremname}{Theorem}

\numberwithin{equation}{subsection}
\numberwithin{figure}{section}
\numberwithin{table}{section}

\theoremstyle{plain}
\newtheorem{prop}{\protect\propositionname}[subsection]
\newtheorem{thm}[prop]{\protect\theoremname}%thm 
\newtheorem{lem}[prop]{Lemma}%lem with counter inherited from prop
\newtheorem{cor}[prop]{Corollary}%cor with counter inherited from prop
\theoremstyle{definition}
\newtheorem{rem}[prop]{\protect\remarkname}
\newtheorem*{rem*}{\protect\remarkname}
\newtheorem{example}[prop]{\protect\examplename}
\newtheorem{defn}[prop]{\protect\definitionname}
\newtheorem{xca}{\protect\exercisename}[section]


\usepackage{tablefootnote}
\usepackage[many,breakable]{tcolorbox}
\ExplSyntaxOn
\NewDocumentCommand{\tcbwrap}{mm}
 {
  \clist_map_inline:nn { #1 } { \tcolorboxenvironment{##1}{#2} }
 }
\ExplSyntaxOff
\tcbwrap{thm,lem,cor,prop}{
    grow to left by=5pt,grow to right by=5pt,
    boxsep=1pt,left=4pt,right=4pt,
    top=2pt,bottom=2pt, colback=gray!20, sharp corners,enhanced,
    leftrule=3pt, rightrule=0pt, toprule=0pt, bottomrule=0pt,
    breakable=true}
\tcbwrap{rem,defn}{
    grow to left by=5pt,grow to right by=5pt,
    boxsep=1pt,left=4pt,right=4pt,
    top=0pt,bottom=0pt,colback=white,boxrule=0pt,frame hidden,sharp corners,enhanced,borderline west={1.5pt}{0pt}{black!80},
    breakable=true}
\tcbwrap{xca,example}{
    grow to left by=5pt,grow to right by=5pt,
    boxsep=1pt,left=4pt,right=4pt,
    top=0pt,bottom=0pt,colback=white,boxrule=0pt,frame hidden,sharp corners,enhanced,
    borderline west={1.5pt}{0pt}{black!80,dashed},
    breakable=true}



%%%%%%%%%%%%%%%%%%%%%%%%%%%%%%%%
\renewenvironment{align*}{\align}{\endalign}
\renewenvironment{alignat*}{\alignat}{\endalignat}
\renewenvironment{multline*}{\multline}{\endmultline}
\renewenvironment{gather*}{\gather}{\endgather}
\renewenvironment{equation*}{\equation}{\endequation}


\usepackage{color}
\def\white{\color{white}}
\def\black{\color{black}}
\def\red{\color{red}}
\def\green{\color{green}}
\def\blue{\color{blue}}
\def\cyan{\color{cyan}}
\def\magenta{\color{magenta}}
\def\yellow{\color{yellow}}



%%%%%%%%%%%%%%%%%%%   MACROS   %%%%%%%%%%%%%%%%%%

% backslash for left side quotients
\newcommand{\bslash}{\mkern-.5mu\backslash\mkern-2mu}

\newcommand{\bbA}{\mathbb{A}}
\newcommand{\bbR}{\mathbb{R}}
\newcommand{\bbN}{\mathbb{N}}
\newcommand{\bbC}{\mathbb{C}}
\newcommand{\bbF}{\mathbb{F}}
\newcommand{\bbZ}{\mathbb{Z}}
\newcommand{\bbH}{\mathbb{H}}
\newcommand{\bbO}{\mathbb{O}}
\newcommand{\bbK}{\mathbb{K}}
\newcommand{\bbP}{\mathbb{P}}
\newcommand{\bbT}{\mathbb{T}}
\newcommand{\bbQ}{\mathbb{Q}}
\newcommand{\bbD}{\mathbb{D}}
\newcommand{\bbS}{\mathbb{S}}

\newcommand{\calA}{\mathcal{A}}
\newcommand{\calC}{\mathcal{C}}
\newcommand{\calD}{\mathcal{D}}
\newcommand{\calF}{\mathcal{F}}
\newcommand{\calG}{\mathcal{G}}
\newcommand{\calL}{\mathcal{L}}
\newcommand{\calJ}{\mathcal{J}}
\newcommand{\calE}{\mathcal{E}}
\newcommand{\calH}{\mathcal{H}}
\newcommand{\calU}{\mathcal{U}}
\newcommand{\calV}{\mathcal{V}}
\newcommand{\calM}{\mathcal{M}}
\newcommand{\calO}{\mathcal{O}}
\newcommand{\calP}{\mathcal{P}}
\newcommand{\calS}{\mathcal{S}}
\newcommand{\calR}{\mathcal{R}}

\newcommand{\fraka}{\mathfrak{a}}
\newcommand{\frake}{\mathfrak{e}}
\newcommand{\frakg}{\mathfrak{g}}
\newcommand{\frakh}{\mathfrak{h}}
\newcommand{\frakk}{\mathfrak{k}}
\newcommand{\fraki}{\mathfrak{i}}
\newcommand{\frakn}{\mathfrak{n}}
\newcommand{\frakm}{\mathfrak{m}}
\newcommand{\frakb}{\mathfrak{b}}
\newcommand{\frakd}{\mathfrak{d}}
\newcommand{\fraku}{\mathfrak{u}}
\newcommand{\frakgl}{\mathfrak{gl}}
\newcommand{\fraksl}{\mathfrak{sl}}
\newcommand{\frakso}{\mathfrak{so}}
\newcommand{\fraksu}{\mathfrak{su}}
\newcommand{\fraksp}{\mathfrak{sp}}

\newcommand{\scrA}{\mathscr{A}}
\newcommand{\scrF}{\mathscr{F}}
\newcommand{\scrH}{\mathscr{H}}
\newcommand{\scrR}{\mathscr{R}}
\newcommand{\scrT}{\mathscr{T}}

\newcommand{\acts}{\curvearrowright}
\newcommand{\lacts}{\curvearrowleft}
\newcommand{\dd}{{\mathrm{d}}}
\newcommand{\bm}[1]{\boldsymbol{\mathrm{#1}}}
\renewcommand{\geq}{\geqslant}
\renewcommand{\leq}{\leqslant}
\renewcommand{\i}{\mathrm{i}}
\renewcommand{\Re}{\operatorname{Re}}
\renewcommand{\Im}{\operatorname{Im}}
\newcommand{\Ad}{\mathrm{Ad}}
\newcommand{\ad}{\mathrm{ad}}
\newcommand{\Adg}{\mathrm{\mathbf{Ad}}}

\newcommand{\rmA}{\mathrm{A}}
\newcommand{\rmP}{\mathrm{P}}
\newcommand{\rmD}{\mathrm{D}}
\newcommand{\rmF}{\mathrm{F}}
\newcommand{\rmG}{\mathrm{G}}
\newcommand{\rmV}{\mathrm{V}}
\newcommand{\rmB}{\mathrm{B}}
\newcommand{\rmE}{\mathrm{E}}
\newcommand{\rmK}{\mathrm{K}}
\newcommand{\rmT}{\mathrm{T}}
\newcommand{\rmi}{\mathrm{i}}
\newcommand{\rme}{\mathrm{e}}

\newcommand{\const}{\mathrm{const}}
\newcommand{\tra}{\mathrm{tra}}
\newcommand{\Lin}{\bm\mathrm{L}}

\newcommand{\sfg}{\mathsf{g}}
\newcommand{\sfh}{\mathsf{h}}
\newcommand{\sfv}{\mathsf{v}}
\newcommand{\sfw}{\mathsf{w}}
\newcommand{\sfJ}{\mathsf{J}}
\newcommand{\sfR}{\mathsf{R}}
\newcommand{\sfT}{\mathsf{T}}

\newcommand{\mFB}{\mathsf{-Bun}}
\newcommand{\FB}{\mathsf{Bun}}
\newcommand{\PFB}{\mathsf{Prin}}
\newcommand{\VB}{\mathsf{VectBun}}
% \DeclareMathOperator{\ad}{ad}
% \DeclareMathOperator{\Ad}{Ad}
% \DeclareMathOperator{\Adg}{\mathbf{Ad}}
\DeclareMathOperator{\sign}{sign}
\DeclareMathOperator{\ver}{ver}
\DeclareMathOperator{\hor}{hor}
\DeclareMathOperator{\mor}{Mor}
\DeclareMathOperator{\rel}{rel\,}
\DeclareMathOperator{\id}{id}
\DeclareMathOperator{\ob}{Ob}
\DeclareMathOperator{\Der}{Der}
\DeclareMathOperator{\Inn}{Inn}
\DeclareMathOperator{\Mat}{Mat}
\DeclareMathOperator{\Aut}{Aut}
\DeclareMathOperator{\aut}{aut}
\DeclareMathOperator{\Gau}{Gau}
\DeclareMathOperator{\gau}{gau}
\DeclareMathOperator{\Fr}{Fr}
\DeclareMathOperator{\End}{End}
\DeclareMathOperator{\Int}{int}
\DeclareMathOperator{\im}{im}
\DeclareMathOperator{\rank}{rank}
\DeclareMathOperator{\pr}{pr}
\DeclareMathOperator{\supp}{supp}
\DeclareMathOperator{\Hom}{Hom}
\DeclareMathOperator{\Ind}{Ind}
\DeclareMathOperator{\Res}{Res}
\DeclareMathOperator{\Diff}{Diff}
\DeclareMathOperator{\Pin}{Pin}
\DeclareMathOperator{\Spin}{Spin}
\DeclareMathOperator{\Hol}{Hol}
\DeclareMathOperator{\hol}{\mathfrak{hol}}
\DeclareMathOperator{\Mon}{Mon}
\DeclareMathOperator{\Aff}{Aff}
\DeclareMathOperator{\GL}{GL}
\DeclareMathOperator{\PGL}{PGL}
\DeclareMathOperator{\SL}{SL}
\DeclareMathOperator{\PSL}{PSL}
\DeclareMathOperator{\Or}{O}
\DeclareMathOperator{\SO}{SO}
\DeclareMathOperator{\CO}{CO}
\DeclareMathOperator{\U}{U}
\DeclareMathOperator{\SU}{SU}
\DeclareMathOperator{\Sp}{Sp}
\DeclareMathOperator{\Tr}{Tr}
\DeclareMathOperator{\tr}{tr}
\DeclareMathOperator{\codim}{codim}
\DeclareMathOperator{\Alt}{Alt}
\DeclareMathOperator{\Sym}{Sym}
\DeclareMathOperator{\grad}{grad}
\DeclareMathOperator{\coker}{coker}
\DeclareMathOperator{\coim}{coim}
\DeclareMathOperator{\dist}{dist}
\DeclareMathOperator{\Ext}{Ext}
\DeclareMathOperator{\ext}{ext}
\DeclareMathOperator{\Tor}{Tor}
\DeclareMathOperator{\tor}{tor}
\DeclareMathOperator{\lcm}{lcm}
\let\div\relax
\DeclareMathOperator{\div}{div}
\DeclareMathOperator*{\res}{res}
\DeclareMathOperator{\Lief}{Lie}

\newcommand{\Lie}{\mathcal{L}}
\newcommand{\colimit}{\varinjlim}
\newcommand{\limit}{\varprojlim}
\newcommand{\fX}{\mathfrak{X}}
\newcommand{\sub}{\vartriangleleft}
\newcommand{\sube}{\trianglelefteqslant}
\newcommand{\<}{\langle}
\renewcommand{\>}{\rangle}
\newcommand{\bigslant}[2]{{\raisebox{.2em}{$#1$}\left/\raisebox{-.2em}{$#2$}\right.}}

\usepackage{xparse}
\DeclareDocumentCommand{\restr}{m m o}  
{%
  \IfNoValueTF{#3}
  {
  \left.\kern-\nulldelimiterspace % automatically resize the bar with \right
  #1 % the function
  \vphantom{\big|} % pretend it's a little taller at normal size
  \right|_{#2}
  }{
  \left.\kern-\nulldelimiterspace % automatically resize the bar with \right
  #1 % the function
  \vphantom{\big|} % pretend it's a little taller at normal size
  \right|_{#2}^{#3} }
}

\makeatletter  %define \xoverline, automatically sized overline
    \newsavebox\myboxA
    \newsavebox\myboxB
    \newlength\mylenA
    \newcommand*\xoverline[2][0.75]{%
        \sbox{\myboxA}{$\m@th#2$}%
        \setbox\myboxB\null% Phantom box
        \ht\myboxB=\ht\myboxA%
        \dp\myboxB=\dp\myboxA%
        \wd\myboxB=#1\wd\myboxA% Scale phantom
        \sbox\myboxB{$\m@th\overline{\copy\myboxB}$}%  Overlined phantom
        \setlength\mylenA{\the\wd\myboxA}%   calc width diff
        \addtolength\mylenA{-\the\wd\myboxB}%
        \ifdim\wd\myboxB<\wd\myboxA%
        \rlap{\hskip 0.5\mylenA\usebox\myboxB}{\usebox\myboxA}%
        \else
            \hskip -0.5\mylenA\rlap{\usebox\myboxA}{\hskip 0.5\mylenA\usebox\myboxB}%
        \fi}
\makeatother

\usepackage{graphicx}
\newcommand{\PRLsep}{   %decorative line used to separate sections
           \noindent\makebox[\linewidth]{
                \resizebox{0.5\linewidth}{1pt}{$\blacklozenge$}}}

\usepackage{graphicx,scalerel}
\newcommand\wh[1]{\hstretch{1.42}{\hat{\hstretch{.7}{#1\mkern2mu}}}\mkern-2mu} %requires \usepackage{graphicx,scalerel}
\newcommand\wt[1]{\hstretch{1.42}{\tilde{\hstretch{.7}{#1\mkern2mu}}}\mkern-2mu}
\newcommand\wb[1]{\hstretch{1.42}{\bar{\hstretch{.7}{#1\mkern2mu}}}\mkern-2mu}



\usepackage[acronym]{glossaries}
\makeglossaries
% abbreviations:
\newacronym{hlp}{HLP}{Homotopy Lifting Property}
\newacronym{hep}{HEP}{Homotopy Extension Property}
\newacronym{pou}{POU}{partition of unity}
\newacronym{fb}{FB}{fiber bundle}
\newacronym{vb}{VB}{vector bundle}
\newacronym{pfb}{PFB}{principal fiber bundle}
\newacronym{wlog}{WLOG}{without loss of generality}
\newacronym{tfae}{TFAE}{the following are equivalent}
\newacronym{inmt}{InMT}{Inverse Mapping Theorem}
\newacronym{immt}{ImMT}{Implicit Mapping Theorem}
\newacronym{mv}{MV}{Mayer-Vietoris}
\newacronym{ops}{OPS}{one-parameter subgroup}


